% 必须使用 XeLaTeX 或 LuaLaTeX 编译
\documentclass{sdu_exp_report}

% 报告抬头信息
\DepartmentName{子虚乌有}
\CourseName{学术理论胡诌学和代码混淆艺术}

% 实验/学生信息 (原始文件中的内容)
\StudentId{202234567890} 
\StudentName{🥬 Lee}
\ClassName{19级810班}

\ExperimentID{1}

\ExperimentTitle{
    防御性编程实操
}
\ExperimentHours{
    2
}
\ExperimentDate{
    \zhtoday  % 使用中文日期格式
}
\ExperimentPurpose{
    熟悉如何假装自己在做学术研究,并通过代码混淆,避免他人理解和复现,来掩盖自己的无能,从而在学术界立足。
}
\HardwareEnv{
    % \input{hardware_env.yml}
    假装这里有
}
\SoftwareEnv{
    % \input{software_env.txt}
    假装这里有
}
\begin{document}

\DrawExperimentReport

\begin{procedure}
这里填写实验步骤和内容。可以使用多行文本。

\begin{enumerate}
    \item 第一步:准备实验环境
    \item 第二步:执行实验操作
    \item 第三步:记录实验数据
    \item 第四步:分析实验结果
\end{enumerate}

如果内容太长,表格会自动扩展。
\end{procedure}

\begin{reflection}
这里填写结论分析与体会。同样支持多行文本。

通过本次实验,我深刻理解了防御性编程的重要性,学会了如何通过代码混淆来保护知识产权(误)。

实验过程中遇到的主要问题及解决方案:
\begin{itemize}
    \item 问题1:代码可读性太高
    \item 解决方案:添加更多混淆逻辑
\end{itemize}
\end{reflection}


\end{document}